\documentclass[aspectratio=169]{beamer}
\usepackage{polski}
\usetheme[lang=pl,hr=true]{NewPwr}
\author{Piotr Łukasiewicz}
\title{Moja pierwsza prezentacja}
\subtitle{ILAAN01}
\institute{Politechnika Wrocławska}
\date{\today}  % automatycznie wstawi datę kompilacji
\begin{document}
\begin{frame} % Slajd tytułowy
 \maketitle
\end{frame}

%Przeźrocze techniczne

\begin{frame}{Zaliczenie zadania}
    \textbf{Link do repozytorium ze źródłami:}
    \begin{center}
        \url{https://github.com/CptLama/ilaan01-piotrl.git}
    \end{center}
    
\vspace{0.5cm}

    \textbf{Wykorzystane elementy:}
\begin{itemize}
        \item Wykorzystanie dedykowanego stylu \texttt{NewPwr}.
        \item Podział treści na sekcje (Plan prezentacji).
        \item Środowisko \texttt{enumerate} (Cykl rozkazowy).
        \item Tabela porównawcza (Analogia do kuchni).
        \item Podział slajdu na kolumny (\texttt{columns}).
    \end{itemize}
\end{frame}

%Plan prezentacji

\begin{frame}{Plan prezentacji}
    \tableofcontents
\end{frame}

%Co to jest Cpu

\section{Definicja CPU}

\begin{frame}{Co to jest CPU?}
    \begin{block}{Central Processing Unit}
        CPU (procesor) to \textbf{mózg komputera}. Jest to główny układ scalony odpowiedzialny za wykonywanie instrukcji i obliczeń.
    \end{block}

    \vspace{0.5cm}
    Kiedy działa procesor?
    \begin{itemize}
        \item Gdy klikasz myszką.
        \item Gdy otwierasz przeglądarkę.
        \item Gdy oglądasz filmy na YouTube.
    \end{itemize}
    
    Każda z tych czynności to dla procesora miliardy prostych poleceń do przetworzenia.
\end{frame}

%Cykl rozkazowy procesora

\section{Cykl pracy (Rozkazy)}

\begin{frame}{Cykl rozkazowy procesora}
    Procesor działa w niekończącej się pętli, wykonując 4 podstawowe kroki miliardy razy na sekundę:

    \vspace{0.3cm}
    \begin{enumerate}
        \item \textbf{Pobieranie (Fetch):} \\
        Pobranie instrukcji z pamięci RAM.
        
        \item \textbf{Dekodowanie (Decode):} \\
        Rozszyfrowanie co należy zrobić (np. "dodaj X do Y").
        
        \item \textbf{Wykonywanie (Execute):} \\
        Faktyczne obliczenie przy użyciu jednostki arytmetyczno-logicznej (ALU).
        
        \item \textbf{Zapisywanie (Store):} \\
        Zapisanie wyniku w pamięci lub rejestrze.
    \end{enumerate}
\end{frame}

%Analogia do kuchni

\section{Analogia (Kuchnia)}

\begin{frame}{Jak to zrozumieć? Analogia kuchenna}
    Wyobraźmy sobie komputer jako kuchnię, a procesor jako kucharza.
    
    \begin{table}[]
        \centering
        \begin{tabular}{ll}
            \toprule
            \textbf{Element w kuchni} & \textbf{Odpowiednik w komputerze} \\
            \midrule
            Kucharz & \textbf{Procesor (CPU)} \\
            Przepis & Program / Kod \\
            Składniki & Dane (np. plik wideo) \\
            Stół kuchenny & Pamięć RAM \\
            Szef kuchni & System operacyjny (OS) \\
            \bottomrule
        \end{tabular}
        \caption{Porównanie pracy komputera do gotowania}
    \end{table}

    \footnotesize
    \textit{Wniosek: Szybszy kucharz (taktowanie) = szybszy posiłek. Więcej rąk do pracy (rdzenie) = więcej dań naraz.}
\end{frame}

%Podsumowanie

\section{Podsumowanie}

\begin{frame}{Podsumowanie}
    \begin{columns}
        \column{0.6\textwidth}
        Podsumowując, CPU to najważniejszy element komputera, który:
        \begin{itemize}
            \item Działa sekwencyjnie (Pobierz $\to$ Dekoduj $\to$ Wykonaj $\to$ Zapisz).
            \item Zarządza wszystkimi zadaniami, od pisania tekstu po gry 3D.
            \item Ściśle współpracuje z pamięcią RAM.
        \end{itemize}
        
        \column{0.4\textwidth}
        \centering
        \textbf{Mózg, który nigdy nie śpi!}
    \end{columns}
\end{frame}

%Koniec

\begin{frame}{Materiały źródłowe}
    Prezentację przygotowano w oparciu o:
    \begin{enumerate}
        \item Wykład "Technologie informacyjne".
        \item Materiały własne dotyczące budowy CPU.
        \item Dokumentacja pakietu \texttt{beamer} oraz styl PWr.
    \end{enumerate}
\end{frame}

\begin{frame}
    \centering
    \Huge \textbf{Dziękuję za uwagę!}
\end{frame}

\end{document}